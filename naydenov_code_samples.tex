\documentclass[11pt, a4paper]{article}
\hoffset=-1in
\voffset=-1in
\textwidth=500pt
\textheight=720pt

\usepackage[T2A, T1]{fontenc}
\usepackage[utf8]{inputenc}
\usepackage[english, russian]{babel}
\usepackage{minted}
\usepackage[nottoc, numbib]{tocbibind}
\usepackage{hyperref}

\begin{document}

\title{Примеры кода}
\author{Алексей Найденов}
\date{Июль 2016}
\maketitle
\tableofcontents

\section{Поиск строк в данных, разделенных на пакеты}

Одной из проблем при разработке системы был поиск заранее заданного
набора строк в данных, которые разбиты на блоки. При этом были
следующие требования к программе: 

\begin{itemize} 
\item Данные разбиты на блоки (пакеты) произвольным образом. Обработку
  необходимо произвести за один проход.
\item Минимизировать объем сохраняемой информации.
\item Строки заданы в момент компиляции, но набор может меняться.
\item Необходимо добиться наибольшей производительности.
\item Строки начинаются с фиксированного разделителя. Эту особенность
  можно использовать для увеличения скорости работы.
\end{itemize}

Для решения задачи была использована машина состояний. Т.к. набор
правил мог меняться в зависимости от требований, то код генерируется
из описания правил при помощи python модуля cog. Для ускорения
переходов между состояниями используется jump table и переходы на
метки. Т.к. строки имеют одинаковые префиксы, то используются
векторные инструкции для обнаружения начала. Поиск происходит за один
проход, для хранения состояния используется 2 байта.

В примере приведены правила для обнаружения ``Content-Type'' в http
заголовке. Т.к. ``octet-stream'' может содержать различный тип данных,
то используется вложенное сравнение. Поиск останавливается при
обнаружении двойного перевода строки.

\subsection{Пример кода.}
\inputminted{cpp}{code/http_media.cc}

\subsection{Программа генерации.}
\inputminted{python}{code/code_generation.py}

\section{Разбор разделенного на пакеты tls заголовка}

Функция для разбора tls заголовка и выделения необходимой информации,
которая способна прерывать и продолжать работу на границе
пакетов. Т.к. заголовок представляет собой вложенную структуру с
полями и массивами переменной длинны, то для хранения состояния
применяется стек. Для ускорения переходов между состояниями
используется jump table. Макросы позволяют улучшить читаемость кода.

\inputminted{cpp}{code/tls_reader.h}
\inputminted{cpp}{code/tls_reader.cc}

\section{Hopscotch hash table с поддержкой ttl}

Реализация hopscotch hash table \cite{hopscotch08}. Добавлена
поддержка удаления элементов, к которым не было обращения в течении
заданного промежутка времени.

\inputminted{cpp}{code/hopscotch_ttl_hash.h}

\section{Lock-free hash table}

Реализация расширяемой lock-free hash table \cite{splitordered06}.

\inputminted{cpp}{code/split_ordered.h}

\section{Lock-free object allocator}

Простой allocator объектов фиксированной длины на основе lock-free
очереди \cite{queue96}. Используется в lock-free hash table.

\inputminted{cpp}{code/object_allocator.h}

\begin{thebibliography}{9}

\bibitem{hopscotch08}
  Maurice Herlihy and Nir Shavit and Moran Tzafrir,
  \emph{Hopscotch Hashing},
  Proceedings of the 22nd international symposium on Distributed Computing,
  Springer-Verlag,
  2008

\bibitem{splitordered06}
  Ori Shalev and Nir Shavit,
  \emph{Split-Ordered Lists: Lock-Free Extensible Hash Tables},
  Journal of the ACM, Vol. 53, No. 3,
  2006.

\bibitem{queue96}
  Maged M. Michael and Michael L. Scott,
  \emph{Simple, fast, and practical non-blocking and
    blocking concurrent queue algorithms},
  Symposium on Principles of Distributed Computing,
  1996.

\end{thebibliography}
\end{document}
